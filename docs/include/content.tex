\chapter{Introduction} 
    \autocite[vgl.][S.245]{01} \\
    \begin{figure}[h]
        \centering
        \includegraphics{images/dhbw-logo.jpg}
        \caption{Meine Grafik}
        \label{fig:meine-grafik}
    \end{figure}
  
\chapter{Requirements}    

\chapter{Prerequisites}
    The following steps and guides relate to docker container, therefore the following preparations have to be made first. 

    \begin{enumerate}
        \item Install \href{https://www.docker.com/products/docker-desktop/}{Docker Desktop}
        \item Install \href{https://code.visualstudio.com/download}{Visual Studio Code} 
        \item Install the VSCode extension \href{https://marketplace.visualstudio.com/items?itemName=ms-vscode-remote.vscode-remote-extensionpack}{Remote Development} 
    \end{enumerate}    

\chapter{How to set up Git for remote development}
    This chapter is about setting up git in a Docker container. When there is no suitable container running yet, then you have to start a new one. This would be the default case. \\
    You have to open a command line interface of your choice (e.g. Powershell for Windows or Bash for Linux). 
    Afterwards you can go through these required steps: 
    \begin{enumerate}
        \item This command will create a container based on the image \textit{ubuntu} with the tag \textit{focal} and afterwards a bash will be opened. You need the options -i and -t for interactive processes like a shell.  
            \begin{lstlisting}[language=bash] 
docker run -it ubuntu:focal bash 
            \end{lstlisting}
        \item APT is a commandline package manager for Linux-distributions and provides commands for searching and managing as well as querying information about packages. You are able to install and uninstall packages with APT. \\
        First the package lists are re-read, afterwards git is installed. 
            \begin{lstlisting}[language=bash] 
apt update && apt install git
            \end{lstlisting}
            You can check wheather the installtion of git was successful with: 
            \begin{lstlisting}[language=bash] 
git - -version
            \end{lstlisting}
        \item GPG is a cryptographic software suite with which you can create key pairs (public and private) and for example create signatures. We will use it to sign our commits on GitHub.
            \begin{lstlisting}[language=bash] 
apt update && apt install gpg
            \end{lstlisting}
        \item Now we will install the GitHub CLI in our container. \\
        When you configure your container from scratch, you have to execute the commands which are provided on \url{https://github.com/cli/cli/blob/trunk/docs/install_linux.md#debian-ubuntu-linux-raspberry-pi-os-apt}
        \item Create a personal access token on \href{https://github.com/settings/tokens}{Github} at least with the permissions read:org, repo and workflow. Copy the generated token and insert the following command in bash with the substituted token: 
            \begin{lstlisting}[language=bash] 
export GH_TOKEN = <your-token>
            \end{lstlisting}   
        Now you are authenticated with your GitHub host and are able to proceed with the next step.
        \item Afterwards you are able to authenticate with your GitHub host via the GitHub CLI. You have multiple choices to do that.  
            \begin{lstlisting}[language=bash] 
gh auth login
            \end{lstlisting}   
        \item Clone the respective GitHub respository with: 
            \begin{lstlisting}[language=bash] 
gh repo clone <link-to-repo>  
            \end{lstlisting} 
            In this case the link to the repo is \textit{themaxens/ball-bearings-with-quarkus}.
        \item Configure git, to create signed commits. Replace <user-name> and <email> with your own credentials. 
            \begin{lstlisting}[language=bash] 
git config gpg.program gpg && git config commit.gpgsign true
git config user.name "<user-name>"
git config user.email "<email>"
            \end{lstlisting} 
    \end{enumerate}

    


\chapter{How to set up Quarkus for remote development}


\chapter{Implementation}


\chapter{Conclusion} % Fazit

